\subsection{elfhash}
如果最高的四位不为0,则说明字符多余7个,现在正在存第8个字符,如果不处理,再加下一个字符时,第一个字符会被移出,因此要有如下处理。 \\
该处理,如果对于字符串(a-z 或者A-Z)就会仅仅影响5-8位,否则会影响5-31位,因为C语言使用的算数移位 \\
因为1-4位刚刚存储了新加入到字符,所以不能右移28 \\
上面这行代码并不会对X有影响,本身X和hash的高4位相同,下面这行代码即对28-31(高4位)位清零。\\
返回一个符号位为0的数,即丢弃最高位,以免函数外产生影响。(我们可以考虑,如果只有字符,符号位不可能为负)\\
hash左移4位,把当前字符ASCII存入hash低四位。 \\
		\begin{lstlisting}[language=c++]
unsigned int ELFHash(char *str)
{
	unsigned int hash = 0;
	unsigned int x = 0;

	while (*str)
	{
		hash = (hash << 4) + (*str++); 
		if ((x = hash & 0xF0000000L) != 0)
		{
			hash ^= (x >> 24);	
			hash &= ~x;
		}
	}
	return (hash & 0x7FFFFFFF);
}
		\end{lstlisting}