\subsection{manacher}
最长回文子串模板\\
hdu3068,最长回文子串模板,Manacher算法,时间复杂度O(n),相当快\\
str是这样一个字符串(下标从1开始):\\
举例:若原字符串为"abcd",则str为"\$\#a\#b\#c\#d\#",最后还有一个终止符。\\
n为str的长度,若原字符串长度为nn,则n=2*nn+2。\\
rad[i]表示回文的半径,即最大的j满足str[i-j+1...i] = str[i+1...i+j],\\
而rad[i]-1即为以str[i]为中心的回文子串在原串中的长度\\

		\begin{lstlisting}[language=c++]
#define M 20000050
char str1[M],str[2*M];//start from index 1
int rad[M],nn,n;
void Manacher(int *rad,char *str,int n)
{
    int i;
    int mx = 0;
    int id;
    for(i=1; i<n; i++)
    {
        if( mx > i )  rad[i] = rad[2*id-i]<mx-i?rad[2*id-i]:mx-i;        
        else rad[i] = 1;
        for(; str[i+rad[i]] == str[i-rad[i]]; rad[i]++);
        if( rad[i] + i > mx )
        {
            mx = rad[i] + i;
            id = i;
        }
    }
}
struct PLD{
	int l,r;
	PLD(int x=0,int y = -1):l(x),r(y){}
}p[N];

void getlr(int n){
	fr(i,2,n){
		p[i].l = i-rad[i]+1;
		p[i].l = (p[i].l+1)/2-1;
		p[i].r = p[i].l+rad[i]-2;
	}
}


int main()
{
	int i,ans,Case=1;
	while(scanf("%s",str1)!=EOF)
	{
		nn=strlen(str1);
		n=2*nn+2;
		str[0]='$';
		for(i=0;i<=nn;i++)
		{
			str[2*i+1]='#';
			str[2*i+2]=str1[i];
		}
		Manacher(rad,str,n);
		ans=1;
		for(i=0;i<n;i++)
			ans=rad[i]>ans?rad[i]:ans;
		printf("%d\n",ans-1);
	}
return 0;
}
		\end{lstlisting}