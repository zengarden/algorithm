\subsection{Dinic}
		\begin{lstlisting}[language=c++]
const int pN=2000,eN=3000000;
struct Edge{
	int u,v,nxt;
	int w;
}e[eN];

int en,head[pN];

void init(){
	memset(head,-1,sizeof(head));
	en=0;
}	
void add(int u,int v,int w){
	e[en].u=u;e[en].v=v;e[en].w=w;e[en].nxt=head[u];head[u]=en++;
	e[en].u=v;e[en].v=u;e[en].w=0;e[en].nxt=head[v];head[v]=en++;
}

int cur[pN],sta[pN],dep[pN];     
int max_flow(int n,int s,int t){   
	int tr,flow = 0;
	int i,u,v,f,r,top;  //f即是front 队列的头部
	int j;
	while(1){
		memset(dep,-1,n*sizeof(int));
		for( f = dep[ sta[0] = s ] = 0 ,r = 1;f != r;){
			for( u = sta[f++],i = head[u];i != -1;i = e[i].nxt){
				if (e[i].w && dep[ v = e[i].v] == -1){
					dep[v] = dep[u] +1;
					sta[r ++] = v; //将v入队列  向后标号法
					if (v == t){
						f = r;
						break;
					}
				}	
			}
		}
		if (-1 == dep[t]) break;
		memcpy(cur,head,n*sizeof(int));
		for (i = s,top = 0; ;){
			if (i == t){
				for( j =0 , tr = inf; j < top; ++j){  //找出一条增广路的最小边权
					if (e[ sta[j] ].w < tr){
						tr = e[ sta[ f = j ] ].w;  //一个简单优化每一次不用从头开始找增广
					}
				}
				for( j = 0;j < top; ++j){
					e[ sta[j] ].w -= tr;
					e[ sta[j]^1].w += tr;
				}
				flow += tr;
				i = e[ sta[top = f] ].u;
			}
			for(j = cur[i]; cur[i] != -1; j = cur[i] = e[cur[i]].nxt)  //i为当前的栈顶元素
				if (e[j].w && dep[i] +1 == dep[e[j].v]) break;  //找到了一条路径最短的增广边
			
			if (cur[i] != -1){   //就是这个点还有出度
				sta[ top++ ] = cur[i];
				i = e[ cur[i] ].v;
			}
			else{
				if (top == 0) break;   
				dep[i] = -1;           
				i = e[sta[--top]].u;  
			}
		}	
	}
	return flow;
}
	\end{lstlisting}