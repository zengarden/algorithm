\subsection{费用流}
		\begin{lstlisting}[language=c++]
const int inf = 0xffffff;
#define M 200001
#define maxx 2000
class Mcmf{
public:
	struct T{
		int u, v, w;
		int nxt, cost;
	}edge[M];
	int en;
	int visit[M], pre[M], dist[M], que[M], vis[M], pos[M];
	void init(){
		memset(vis,-1,sizeof(vis));
		en=0;
	}
	void add(int u, int v, int w, int cost)  
	{
		edge[en].u = u,edge[en].v = v, edge[en].w = w, edge[en].cost = cost;
		edge[en].nxt = vis[u], vis[u] = en++;
		edge[en].u= v, edge[en].v = u, edge[en].w = 0, edge[en].cost = -cost;
		edge[en].nxt = vis[v], vis[v] = en++;
	}
	bool spfa(int n,int s,int t){
		int v,k;
		for (int i = 0;i <= n; i++){
			pre[i] = -1,visit[i] = 0;
		}
		int f = 0,r = 0;	
		for (int i = 0;i <= n; ++i) dist[i] = -1;
		que[r ++] = s;pre[s] = s;dist[s] = 0;visit[s] = 1;
		while(f != r){
			int u = que[f ++];
			visit[u] = 0;
			for (k = vis[u] ;k != -1;k = edge[k].nxt){   
				v = edge[k].v;
				if (edge[k].w && dist[u] + edge[k].cost > dist[v]){
					dist[v] = dist[u] + edge[k].cost;
					pre[v] = u;
					pos[v] = k;   //是哪一条边到大的v 巧妙呀 值得学习一下~~~
					if (! visit[v]){
						visit[v] = 1;
						que[r ++] = v;
					}
				}
			}
		}
		if (pre[t] != -1 &&dist[t] > -1) return 1;
		return 0;
	}
	int mnCostFlow(int n,int s,int t){
		if (s == t){}
		int flow =0,cost =0;
		while(spfa(n,s,t)){
			int u,mn = inf;
			for( u = t;u != s; u = pre[u]) 
				if (mn > edge[pos[u]].w) mn = edge[pos[u]].w;
			flow += mn;
			cost += dist[t] * mn;
			for(u = t;u != s;u = pre[u]){
				edge[pos[u]].w -= mn;
				edge[pos[u]^1].w += mn;
			}
		}
		return cost;
	}
}mcf;
	\end{lstlisting}