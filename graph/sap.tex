\subsection{sap}
这里是与dinic不同的地方不用每次的bfs 而是充分利用以前的距离标号的信息
有这个定理:从源点到汇点的最短路一定是用允许弧构成。 
所以每次扩展路径都找允许弧,
如果i没有允许弧就更新dis[i] = min  dis[j] + 1 或者 r[i][j] 大于 0) ;
		\begin{lstlisting}[language=c++]
#define inf 1000000000
using namespace std;

const int pN=5000,eN=100000;

struct Edge{
	int u,v,nxt;
	int w;
}e[eN];

int en,head[pN];

void init(){
	memset(head,-1,sizeof(head));
	en=0;
}
void add(int u,int v,int w){
	e[en].u=u;e[en].v=v;e[en].w=w;e[en].nxt=head[u];head[u]=en++;
	e[en].u=v;e[en].v=u;e[en].w=0;e[en].nxt=head[v];head[v]=en++;
}
int dep[pN],gap[pN],que[pN]; //gap  每一次重标号时 若出现了断层,则可以证明st无可行流,此时可以直接退出算法
void BFS(int n,int s,int t){
	memset(dep,-1,n * sizeof(int));
	memset(gap, 0 ,n * sizeof (int));
	gap[0] = 1;
	int f = 0,r = 0,u,v;
	dep[ t ] = 0; que[r ++] = t; //从后外前面标号
	while(f != r){
		u = que[f ++];
		if ( f == pN) f = 0;
		for(int i = head[u];i != -1;i = e[i].nxt){
			v = e[i].v;
			if (e[i].w != 0 || dep[v] != -1) continue;  //如果容量为0  就根本到不到它
			que[ r++ ] = v;
			if (r == pN) r = 0;
			dep[ v ] = dep[ u ] + 1;
			++ gap[dep[ v ]];   //这里的gap就是每一层有多少个点
		}
	}
}

int cur[pN],sta[pN];
int sap(int n,int s,int t){        //n为总的点个数 包括源点和汇点
	int flow = 0;
	BFS(n,s,t);
	int top = 0,u = s,i;       
	memcpy(cur,head,n*sizeof(int));   //当前弧
	while( dep[s] < n){
		if ( u == t){
			int tmp = inf;
			int pos;     
			for(i = 0;i < top;i++){
				if (tmp > e[ sta[i] ].w){
					tmp = e[ sta[i] ].w;
					pos = i;
				}
			}
			for(i = 0;i < top; ++i){
				e[ sta[i] ].w -= tmp;
				e[ sta[i]^1 ].w += tmp;
			}
			flow += tmp;
			top = pos;
			u = e[sta[top]].u;
		}
		if(u != t && gap[dep[u] - 1] == 0)  break;  //gap 优化 出现断层后直接退出
		for(i = cur[u] ; i != -1 ;i = e[i].nxt)          //当前弧优化  因为以前的弧绝对不满足要求
			if(e[i].w != 0 && dep[u] == dep[e[i].v] + 1) break;  //找到了一条最短增广路
		if (i != -1) cur[ u ] = i,sta[top ++] = i, u = e[i].v;   
		else{                                     
			//这里与dinic不同
			int mn = n;
			for (i = head[u] ;i != -1;i = e[i].nxt){     
				if ( e[i].w != 0 && mn > dep[ e[i].v ] ){
					mn = dep[ e[i].v ] ;
					cur[u] = i;                          
				}
			}
			-- gap[ dep[u] ];  
			dep[u] = mn + 1;   
			++ gap[ dep[u] ];  
			if (u != s) u = e[sta[--top]].u;
		}
	}
	return flow;
}
	\end{lstlisting}