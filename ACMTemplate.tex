\documentclass[12pt,a4paper]{article}

\usepackage{makeidx} %制作索引
\usepackage{indentfirst,amssymb}
\usepackage{listings} %用于放代码
\usepackage{pstricks} %Pstricks是非常有效的画图包
\usepackage{pst-node,pst-tree} %
\usepackage{graphicx} %提供了一组控制命令,其中最常用的是插图命令
\usepackage{amsmath}  %美国数学会开发(amsmath宏包)排版的数学公式
\usepackage{type1cm}  %任意大小西文 (要添加这个才行)
\usepackage{xcolor}  %改变字体颜色
\usepackage{fancyhdr} %自定义页眉和页脚。
\usepackage{booktabs} %latex编写表格
\usepackage{geometry} %设置页面边距
\usepackage{xeCJK}

\setCJKmainfont{SimSun}
\setCJKsansfont{SimSun}
\setCJKmonofont{SimSun}
\setCJKsansfont{Source Code Pro}

\geometry{left=3cm,right=1cm,top=1.5cm,bottom=2cm,headsep=0.2cm}
\lstset{
	tabsize=4,
	numbers=left,
	numberstyle=\sf\normalsize,   %\tiny,                            %\sf\normalsize,
	keywordstyle=\color{blue!70},                 %\sf\normalsize,
	commentstyle=\color{red!50!green!50!blue!50}, %\sf\normalsize,
	rulesepcolor=\color{red!20!green!20!blue!20},
	frame=leftline,%shadowbox,	
	escapeinside=``,
    extendedchars=false
}

\usepackage[CJKbookmarks=true,
			colorlinks,
            linkcolor=black,
            anchorcolor=black,
            citecolor=black]{hyperref}
\AtBeginDvi{\special{pdf:tounicode UTF8-UCS2}}

            

\setlength{\parindent}{0em}
\begin{document}

%\begin{CJK*}{UTF8}{gbsn}
	\pagestyle{fancy} %fancyhdr包来设置页眉页脚
	\lhead{} 
    \chead{} 
    \rhead{\bfseries ACM Template}  %\bfseries设置部分文字为黑体字
    \lfoot{\bfseries zengarden} 
    \cfoot{}
    \rfoot{\bfseries\thepage} 
    \renewcommand{\headrulewidth}{0.4pt}  %newcommand 是定义一个系统不存在的命令,用户为了方便自己可以定义便于自己阅读和使用的命令
    \renewcommand{\footrulewidth}{0.4pt} %renewcommand是重定义一个命令, 我们可以把系统的已有的命令进行重定义。
	%\title{ACM Template}
	%\author{黎泽明}
	%\date{2013/06/01}
	%\maketitle    
    
    \newgeometry{left=1cm,right=1cm,top=1.5cm,bottom=1.5cm}
    \begin{titlepage}
    %~ %强制换页
	\clearpage
	\pagestyle{empty}
	
        \begin{center}
        ~\\[160pt]
        
        \hrule\ \\[8pt]
        \fontsize{48pt}{\baselineskip}\selectfont  \textsc{ACM Template}\\[8pt]
        \hrule\ \\[340pt]

        \huge Zengarden\\[8pt]
        \Large Last build at \today
        \end{center}
    \end{titlepage}
    \restoregeometry
    
    \tableofcontents  %生成目录
    \clearpage
    
    \section{数据结构}
	\input datastructure/bit.tex
    \input datastructure/DiscreteCoordinates.tex
	\input datastructure/lca-rmq.tex
	\input datastructure/rmq2d.tex
	\input datastructure/partitiontree.tex
	\input datastructure/scanline.tex
	\clearpage  %\newpage:  The \newpage command ends the current page. \clearpage:The \clearpage 生成新的页
	
	\section{数学}
    \input math/prime.tex
	\input math/Gcd.tex
	\input math/extend-gcd.tex
	\input math/add_mod.tex
	\input math/pow_mod.tex
	\input math/matrix.tex
	\input math/inclusion-exclusion.tex
	\input math/combination.tex
	\input math/pollardRho.tex
	\input math/euler_function.tex
	\input math/gauss.tex
	\input math/gray.tex
	\input math/discrete_log.tex
	\input math/arr2int.tex
	\input math/polya.tex
	
	\clearpage
	
	\section{string}
    \input string/kmp.tex
    \input string/ekmp.tex
    \input string/manacher.tex
    \input string/ac_auto.tex
    \input string/suffixarray.tex
    \input string/sam.tex
    \input string/ELFhash.tex
    \input string/hashtable.tex
    \input string/otherhash.tex
    
	
	\clearpage
	
	\section{图论}
    \input graph/ForwardStar.tex
    \input graph/DisjointSet.tex
    \input graph/spfa.tex
    \input graph/OnlineLca.tex
    \input graph/dinic.tex
	\input graph/sap.tex
	\input graph/mcf.tex
	\clearpage
	
	\section{计算几何}
    \input geometric/DynamicConvex.tex
    
	\clearpage
	
	

%\end{CJK*}
\printindex
\end{document}