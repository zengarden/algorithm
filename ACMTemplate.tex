\documentclass[12pt,a4paper]{article}
\usepackage{indentfirst}
\usepackage{mathpazo}
\usepackage{xeCJK}
\usepackage{graphicx} %提供了一组控制命令,其中最常用的是插图命令
\usepackage{amsmath} %美国数学会开发(amsmath宏包)排版的数学公式
\usepackage{xcolor}  %改变字体颜色
\usepackage{type1cm} %任意大小西文 (要添加这个才行)
\usepackage{booktabs} %latex编写表格
\usepackage{geometry}  %设置页面边距
\usepackage{courier}  %标准的等宽度字体
\usepackage{listings} %用于放代码
\usepackage{fancyhdr}  %自定义页眉和页脚。
\usepackage{pstricks,pst-node,pst-tree} %Pstricks是非常有效的画图包
%\usepackage{time}
%\usepackage{charter}
%\usepackage[landscape]{geometry}

%\setsansfont{Inconsolata}
%\setsansfont{DejaVu Sans Mono}
\setsansfont{Source Code Pro}
%\setsansfont{Monaco}

\setCJKmainfont{SimHei}
\setCJKsansfont{SimHei}
\setCJKmonofont{SimHei}

\geometry{left=3cm,right=1cm,top=1.5cm,bottom=2cm,headsep=0.2cm}


\lstset{
    breaklines=true,
    tabsize=2,
    basicstyle=\sf\normalsize,
    numberstyle=\sf\normalsize,
    commentstyle=\sf\normalsize,
    numbers=left,
    %keywordstyle=\color{blue!70}, 
    frame=leftline,
    %rulesepcolor=\color{red!20!green!20!blue!20},
    escapeinside=``,
    extendedchars=false
}
\usepackage[CJKbookmarks=true,
			colorlinks,
            linkcolor=black,
            anchorcolor=black,
            citecolor=black]{hyperref}
\AtBeginDvi{\special{pdf:tounicode UTF8-UCS2}}
\setlength{\parindent}{0em}

\begin{document}
    \pagestyle{fancy} %fancyhdr包来设置页眉页脚
    \lhead{} 
    \chead{} 
    \rhead{\bfseries ACM Template}  %\bfseries设置部分文字为黑体字
    \lfoot{\bfseries Zengarden} 
    \cfoot{}
    \rfoot{\bfseries\thepage} 
    \renewcommand{\headrulewidth}{0.4pt}  %newcommand 是定义一个系统不存在的命令,用户为了方便自己可以定义便于自己阅读和使用的命令
    \renewcommand{\footrulewidth}{0.4pt}  %renewcommand是重定义一个命令, 我们可以把系统的已有的命令进行重定义。

    \newgeometry{left=1cm,right=1cm,top=1.5cm,bottom=1.5cm}
    \begin{titlepage}
	%~
	\clearpage
	\pagestyle{empty}
	
        \begin{center}
        ~\\[160pt]
        
        \hrule\ \\[8pt]
        \fontsize{48pt}{\baselineskip}\selectfont  \textsc{ACM Template}\\[8pt]
        \hrule\ \\[340pt]

        \huge Zengarden\\[8pt]
        \Large Last build at \today
        \end{center}
    \end{titlepage}

    \restoregeometry
    
    \tableofcontents
    \clearpage

    %\input todolist.tex
    %\clearpage

    \input string.tex
    \clearpage

    \input math.tex
    \clearpage

    \input datastructure.tex
    \clearpage

    \input graph.tex
    \clearpage

    \input geometric.tex
    \clearpage


\end{document}