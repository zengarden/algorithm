\subsection{划分树}
		\begin{lstlisting}[language=c++]
int a[100001];
int b[21][100001];
int sum[21][100001];   //sum[i]表示l--i这些点中有多少个进入了左子树。
int n,m;

void build(int l,int r,int d){    //d代表在树上第几层
	if (l==r) return;
	int i,mid=(l+r)>>1,id1=l,id2=mid+1,midsum=0;
	for (i=mid;i>=l&&a[i]==a[mid];i--)midsum+=1;
	for (i=l;i<=r;i++){
		sum[d][i]=i==l?0:sum[d][i-1];
		if (b[d][i]<a[mid]){
			b[d+1][id1++]=b[d][i];
			sum[d][i]+=1;
		}
		else if(b[d][i]==a[mid]&&midsum){
			midsum-=1;
			b[d+1][id1++]=b[d][i];
			sum[d][i]+=1;
		}
		else b[d+1][id2++]=b[d][i];
	}
	build(l,mid,d+1);
	build(mid+1,r,d+1);
}

int search(int x,int y,int k){
	int l=1,r=n,d=0;
	int ls,rs,mid;
	while (x!=y){
		ls=x==l?0:sum[d][x-1];    //因为要包含x
		rs=sum[d][y];
		mid=(l+r)>>1;
		if (k<=rs-ls) {        //在左子树上
			x=l+ls;
			y=l+rs-1;
			r=mid;
		}
		else                 //在右子树上
		{
			x=mid+1+x-l-ls;  // (x-l-ls)是指处在x前面且进入右子树的个数,因为在子树中保持位置顺序不变,所以在右子树中x前面有(x-l-ls)个数。
			y=mid+1+y-l-rs;
			k-=rs-ls;
			l=mid+1;
		}
		d+=1;
	}
	return b[d][x];
}

int main(){
	freopen("in.txt","r",stdin); 
	int cnt=1;
	while(scanf("%d",&n)!=EOF){
		int i,x,y,t;
		for (i=1;i<=n;i++){
			scanf("%d",&t);
			a[i]=b[0][i]=t;
		}
		sort(a+1,a+n+1);
		build(1,n,0);
		scanf("%d",&m);
		printf("Case %d:\n",cnt++);
		while(m--){
			scanf("%d%d",&x,&y);
			int k=(y-x+1)/2+1;
			printf("%d\n",search(x,y,k));
		}
	}
	return 0;
}
	\end{lstlisting}