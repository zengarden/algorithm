\subsection{组合数}
	\subsubsection{暴力求解}
		\begin{lstlisting}[language=c++]
C(n,m)=n*(n-1)*...*(n-m+1)/m!,n<=15
int Combination(int n, int m)
{
	const int M = 10007;
	int ans = 1;
	for(int i=n; i>=(n-m+1); --i)
		ans *= i;
	while(m)
		ans /= m--;
	return ans % M;
}
		\end{lstlisting}
	\subsubsection{打表}
		\begin{lstlisting}[language=c++]
C(n,m)=C(n-1,m-1)+C(n-1,m),n<=10,000
const int M = 10007;
const int N = 1000;
ll C[N][N];
void initc(){
	int i,j;
	for(i=0; i<N; ++i){
		C[0][i] = 0;
		C[i][0] = 1;
	}
	for(i=1; i<N; ++i){
		for(j=1; j<N; ++j)
			C[i][j] = (C[i-1][j] + C[i-1][j-1]) % MOD;
	}
}
		\end{lstlisting}
	\subsubsection{质因数分解}
	C(n,m)=n!/(m!*(n-m)!),C(n,m)=p1a1-b1-c1p2a2-b2-c2pkak-bk-ck,n<=10,000,000\\
		\begin{lstlisting}[language=c++]
//用筛法生成素数
const int MAXN = 1000000;
bool arr[MAXN+1] = {false};
vector<int> produce_prim_number(){
	vector<int> prim;
	prim.push_back(2);
	int i,j;
	for(i=3; i*i<=MAXN; i+=2){
		if(!arr[i]){
			prim.push_back(i);
			for(j=i*i; j<=MAXN; j+=i)
			arr[j] = true;
		}
	}
	while(i<=MAXN){
		if(!arr[i])
		prim.push_back(i);
		i+=2;
	}
	return prim;
}

//计算n!中素因子p的指数
int Cal(int x, int p){
	int ans = 0;
	long long rec = p;
	while(x>=rec){
		ans += x/rec;
		rec *= p;
	}
	return ans;
}

//计算n的k次方对M取模,二分法
int Pow(long long n, int k, int M){
	long long ans = 1;
	while(k){
		if(k&1){
			ans = (ans * n) % M;
		}
		n = (n * n) % M;
		k >>= 1;
	}
	return ans;
}

//计算C(n,m)
int Combination(int n, int m){
    const int M = 10007;
	vector<int> prim = produce_prim_number();
	long long ans = 1;
	int num;
	for(int i=0; i<prim.size() && prim[i]<=n; ++i){
		num = Cal(n, prim[i]) - Cal(m, prim[i]) - Cal(n-m, prim[i]);
		ans = (ans * Pow(prim[i], num, M)) % M;
	}
	return ans;
}
		\end{lstlisting}
	\subsubsection{Lucas}
	/*定理,将m,n化为p进制,有:C(n,m)=C(n0,m0)*C(n1,m1)...(mod p),算一个不是很大的C(n,m)\%p,p为素数,化为线性同余方程,用扩展的欧几里德定理求解,n在int范围内,修改一下可以满足long long范围内。*/\\
		\begin{lstlisting}[language=c++]	

 
const int M = 10007;
int ff[M+5];  //打表,记录n!,避免重复计算

//求最大公因数
int gcd(int a,int b){
    if(b==0)
		return a;
    else
		return gcd(b,a%b);
}

//解线性同余方程,扩展欧几里德定理
int x,y;
void Extended_gcd(int a,int b){
    if(b==0)    {
       x=1;
       y=0;
    }
    else{
       Extended_gcd(b,a%b);
       long t=x;
       x=y;
       y=t-(a/b)*y;
    }
}

//计算不大的C(n,m)
int C(int a,int b){
    if(b>a)
	return 0;
    b=(ff[a-b]*ff[b])%M;
    a=ff[a];
    int c=gcd(a,b);
    a/=c;
    b/=c;
    Extended_gcd(b,M);
    x=(x+M)%M;
    x=(x*a)%M;
    return x;
}

//Lucas定理
int Combination(int n, int m){
    int ans=1;
	int a,b;
	while(m||n){
	         a=n%M;
		b=m%M;
		n/=M;
		m/=M;
		ans=(ans*C(a,b))%M;
	}
	return ans;
}

int main(void){
    int i,m,n;
	ff[0]=1;
	for(i=1;i<=M;i++)  //预计算n!
	ff[i]=(ff[i-1]*i)%M;
	
	scanf("%d%d",&n, &m);
	printf("%d\n",func(n,m));
	
	return 0;
}
		\end{lstlisting}
