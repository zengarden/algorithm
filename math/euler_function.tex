\subsection{欧拉函数}
	\subsubsection{一般的求法}
		\begin{lstlisting}[language=c++]
const int N = 100010;
bool is_prime[N];
ll phi[N];
ll prime[N];
void init(){
	ll i, j, k = 0;
	phi[1] = 1;
	for(i = 2; i < N; i++){
		if(is_prime[i] == false){
			prime[k++] = i;
			phi[i] = i-1;
		}
		for(j = 0; j<k && i*prime[j]<N; j++){
			is_prime[ i*prime[j] ] = true;
			if(i%prime[j] == 0){
				phi[ i*prime[j] ] = phi[i] * prime[j];
				break;
			}
			else phi[ i*prime[j] ] = phi[i] * (prime[j]-1);
		}
	}
}
		\end{lstlisting}
	\subsubsection{递推}
		\begin{lstlisting}[language=c++]
for (i = 1; i <= maxn; i++) phi[i] = i; 
for (i = 2; i <= maxn; i += 2) phi[i] /= 2; 
for (i = 3; i <= maxn; i += 2) if(phi[i] == i) { 
    for (j = i; j <= maxn; j += i) 
        phi[j] = phi[j] / i * (i - 1); 
}
		\end{lstlisting}
	\subsubsection{单独求}
		\begin{lstlisting}[language=c++]
ll Euler_Phi(ll n)
{
	ll t = n,p = n;
	ll sq = sqrt (n);

	for (int i=0;prime[i]<=sq && i<totpri;i++)
	{
		if (t%prime[i]==0)
		{
			p = p/prime[i]*(prime[i]-1);

			while (t%prime[i]==0)
				t/=prime[i];

			//sq = sqrt(t);
		}

		if (t == 1)
			break;
	}

	if (t > 1)
		p = p/t*(t-1);

	return p;
}
		\end{lstlisting}
	