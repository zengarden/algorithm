\subsection{矩阵类}
		\begin{lstlisting}[language=c++]
ll MOD=10000;
template<int MAXN=1010,int MAXM=1010, typename T = int>
struct Mat{
    int n,m;
    T a[MAXN][MAXM];
    Mat(int _n=0,int _m=0):n(_n),m(_m){}
    void clear(){
        memset(a,0,sizeof(a));
    }
    void identity(){
        memset(a,0,sizeof(a));
        fr(i , 0 ,n){
            a[i][i] = 1;
        }
    }
    Mat operator + (const Mat &b) const{
        Mat tmp(n,m);
        for(int i = 0;i<n;++i){
            for(int j = 0;j<m;++j){
                tmp.a[i][j] = a[i][j] + b.a[i][j];
            }
        }
        return tmp;
    }
    Mat operator - (const Mat &b) const{
        Mat tmp(n,m);
        for(int i = 0;i<n;++i){
            for(int j = 0;j<m;++j){
                tmp.a[i][j] = a[i][j] - b.a[i][j];
            }
        }
        return tmp;
    }
    Mat operator * (const Mat &b) const{
        Mat tmp(n,m);
        tmp.clear();
        for(int i = 0;i<n;++i){
            for(int j = 0;j<n;++j)
                for(int k = 0;k < n;++k){
                tmp.a[i][j] = (tmp.a[i][j] + a[i][k] * b.a[k][j]) % MOD;
            }
        }
        return tmp;
    }
    Mat operator ^ (int b) {
        Mat ret(n,m);
        ret.identity();
        while(b){
            if(b&1)
                ret = (*this)*ret;
            (*this) = (*this)*(*this);
            b>>=1;
        }
        return ret;
    }
    void disp(){
        fr(i , 0 ,n){
            fr(j , 0 ,m){
                printf("%d ",a[i][j]);
            }
            puts("");
        }
    }
};
		\end{lstlisting}